%%%%%%%%%%%%%%%%%%%%%%%%%
% Author: Xingdong Zuo  %
%                       %
% Version: June 21, 2018 %
%%%%%%%%%%%%%%%%%%%%%%%%%


%%%%%%%%%%%%%%%%%%
% Basic setting  %
%%%%%%%%%%%%%%%%%%
\usepackage[utf8]{inputenc} % allow utf-8 input
\usepackage[T1]{fontenc}    % use 8-bit T1 fonts
\usepackage{hyperref}       % hyperlinks
\usepackage{url}            % simple URL typesetting
\usepackage{nicefrac}       % compact symbols for 1/2, etc.
\usepackage{microtype}      % microtypography
\usepackage{authblk}        % Authors and affiliations


%%%%%%%%%%%%%%%%%%%%%%%%%%%%%%%%%%%%%%%%%%%%%%%%%%%%%%%%%%%
% Bibliography management (get item from Google Scholar)  %
%%%%%%%%%%%%%%%%%%%%%%%%%%%%%%%%%%%%%%%%%%%%%%%%%%%%%%%%%%%
% Use: 
%   1. Create a file called 'references.bib'
%   2. Copy items from Google scholar
%   3. \cite{tamar2016value}
%   4. \printbibliography
\usepackage[
style=numeric,      % [numeric, alphabetic]
sorting=nty         % nty: name, title, year
]{biblatex}
\addbibresource{references.bib}

% Figure, tables
\usepackage{graphicx}
\graphicspath{ {img/} }     % search path of images
\usepackage{subcaption}     % Multiple figures \subfigure
\usepackage{booktabs}       % professional-quality tables

% Mathematical
\usepackage{amsmath, amssymb, amsthm, amsfonts}
\usepackage{mathtools}  % more math
\usepackage{bbm}                    % '\mathbbm' for digits
\usepackage{bm}                     % bold symbol for greek letters
\usepackage[ruled]{algorithm2e}     % Algorithms package


%%%%%%%%%%%%%%%%%%%%%%%
% Reference commands  %
%%%%%%%%%%%%%%%%%%%%%%%
\newcommand{\Figref}[1]{Figure~\ref{#1}}  % Figure reference
\newcommand{\Secref}[1]{Section~\ref{#1}}  % Section reference
\newcommand{\Eqref}[1]{Equation~(\ref{#1})}  % Equation reference
\newcommand{\Algoref}[1]{Algorithm~\ref{#1}}  % Algorithm reference
\newcommand{\Defref}[1]{Definition~\ref{#1}}    % Definition reference
\newcommand{\Thmref}[1]{Theorem~\ref{#1}}   % Theorem reference
\newcommand{\Lemref}[1]{Lemma~\ref{#1}}   % Lemma reference
\newcommand{\Cororef}[1]{Corollary~\ref{#1}}    % Corollary reference
\newcommand{\Propref}[1]{Proposition~\ref{#1}}  % Proposition reference


%%%%%%%%%%%%%%%%%%%%%%%%%%%%%%%%%%%%
% Definition, Theorems, proof etc. %
%%%%%%%%%%%%%%%%%%%%%%%%%%%%%%%%%%%%
\newtheorem{definition}{Definition}[section]  % [section]: recount after each section
\newtheorem{theorem}{Theorem}[section]
\newtheorem{lemma}[theorem]{Lemma}            % [theorem]: same counter as theorem
\newtheorem{corollary}[theorem]{Corollary}
\newtheorem{proposition}[theorem]{Proposition}
\newtheorem{hypothesis}[theorem]{Hypothesis}
\newtheorem{conjecture}[theorem]{Conjecture}
\newtheorem{principle}[theorem]{Principle}
\newtheorem{claim}[theorem]{Claim}
\newtheorem{example}[theorem]{Example}
% \begin{proof}\end{proof} already included in package amsthm

\newtheorem{assumption}{Assumption}
\newtheorem{question}{Question}
\newtheorem{remark}{Remark}
\newtheorem{case}{Case}


%%%%%%%%%
% Fonts %
%%%%%%%%%

%%%%%%%%%%%%%%%%%%%%%%%%%%%%%%%%%%%%%%%%%%%%%%%%
% Math bold font: e.g. variable, vector/matrix %
%%%%%%%%%%%%%%%%%%%%%%%%%%%%%%%%%%%%%%%%%%%%%%%%
\newcommand{\bfa}{\mathbf{a}}
\newcommand{\bfb}{\mathbf{b}}
\newcommand{\bfc}{\mathbf{c}}
\newcommand{\bfd}{\mathbf{d}}
\newcommand{\bfe}{\mathbf{e}}
\newcommand{\bff}{\mathbf{f}}
\newcommand{\bfg}{\mathbf{g}}
\newcommand{\bfh}{\mathbf{h}}
\newcommand{\bfi}{\mathbf{i}}
\newcommand{\bfj}{\mathbf{j}}
\newcommand{\bfk}{\mathbf{k}}
\newcommand{\bfl}{\mathbf{l}}
\newcommand{\bfm}{\mathbf{m}}
\newcommand{\bfn}{\mathbf{n}}
\newcommand{\bfo}{\mathbf{o}}
\newcommand{\bfp}{\mathbf{p}}
\newcommand{\bfq}{\mathbf{q}}
\newcommand{\bfr}{\mathbf{r}}
\newcommand{\bfs}{\mathbf{s}}
\newcommand{\bft}{\mathbf{t}}
\newcommand{\bfu}{\mathbf{u}}
\newcommand{\bfv}{\mathbf{v}}
\newcommand{\bfw}{\mathbf{w}}
\newcommand{\bfx}{\mathbf{x}}
\newcommand{\bfy}{\mathbf{y}}
\newcommand{\bfz}{\mathbf{z}}

\newcommand{\bfA}{\mathbf{A}}
\newcommand{\bfB}{\mathbf{B}}
\newcommand{\bfC}{\mathbf{C}}
\newcommand{\bfD}{\mathbf{D}}
\newcommand{\bfE}{\mathbf{E}}
\newcommand{\bfF}{\mathbf{F}}
\newcommand{\bfG}{\mathbf{G}}
\newcommand{\bfH}{\mathbf{H}}
\newcommand{\bfI}{\mathbf{I}}
\newcommand{\bfJ}{\mathbf{J}}
\newcommand{\bfK}{\mathbf{K}}
\newcommand{\bfL}{\mathbf{L}}
\newcommand{\bfM}{\mathbf{M}}
\newcommand{\bfN}{\mathbf{N}}
\newcommand{\bfO}{\mathbf{O}}
\newcommand{\bfP}{\mathbf{P}}
\newcommand{\bfQ}{\mathbf{Q}}
\newcommand{\bfR}{\mathbf{R}}
\newcommand{\bfS}{\mathbf{S}}
\newcommand{\bfT}{\mathbf{T}}
\newcommand{\bfU}{\mathbf{U}}
\newcommand{\bfV}{\mathbf{V}}
\newcommand{\bfW}{\mathbf{W}}
\newcommand{\bfX}{\mathbf{X}}
\newcommand{\bfY}{\mathbf{Y}}
\newcommand{\bfZ}{\mathbf{Z}}

%%%%%%%%%%%%%%%%%%%%%%%%%%%%%%%%%%%%%%%%%%%%%%%%%%%%%%%%%%%%%%
% Blackboard bold: e.g. real/complex set, indicator function %
%%%%%%%%%%%%%%%%%%%%%%%%%%%%%%%%%%%%%%%%%%%%%%%%%%%%%%%%%%%%%%
\newcommand{\bbA}{\mathbb{A}}
\newcommand{\bbB}{\mathbb{B}}
\newcommand{\bbC}{\mathbb{C}}
\newcommand{\bbD}{\mathbb{D}}
\newcommand{\bbE}{\mathbb{E}}
\newcommand{\bbF}{\mathbb{F}}
\newcommand{\bbG}{\mathbb{G}}
\newcommand{\bbH}{\mathbb{H}}
\newcommand{\bbI}{\mathbb{I}}
\newcommand{\bbJ}{\mathbb{J}}
\newcommand{\bbK}{\mathbb{K}}
\newcommand{\bbL}{\mathbb{L}}
\newcommand{\bbM}{\mathbb{M}}
\newcommand{\bbN}{\mathbb{N}}
\newcommand{\bbO}{\mathbb{O}}
\newcommand{\bbP}{\mathbb{P}}
\newcommand{\bbQ}{\mathbb{Q}}
\newcommand{\bbR}{\mathbb{R}}
\newcommand{\bbS}{\mathbb{S}}
\newcommand{\bbT}{\mathbb{T}}
\newcommand{\bbU}{\mathbb{U}}
\newcommand{\bbV}{\mathbb{V}}
\newcommand{\bbW}{\mathbb{W}}
\newcommand{\bbX}{\mathbb{X}}
\newcommand{\bbY}{\mathbb{Y}}
\newcommand{\bbZ}{\mathbb{Z}}

%%%%%%%%%%%%%%%%%%%%%%%%%%%%%%%%%%%%%%%%%%%%%%%%%%%%%%%
% Calligraphic font: e.g. space, relations, power set %
%%%%%%%%%%%%%%%%%%%%%%%%%%%%%%%%%%%%%%%%%%%%%%%%%%%%%%%
\newcommand{\ccA}{\mathcal{A}}
\newcommand{\ccB}{\mathcal{B}}
\newcommand{\ccC}{\mathcal{C}}
\newcommand{\ccD}{\mathcal{D}}
\newcommand{\ccE}{\mathcal{E}}
\newcommand{\ccF}{\mathcal{F}}
\newcommand{\ccG}{\mathcal{G}}
\newcommand{\ccH}{\mathcal{H}}
\newcommand{\ccI}{\mathcal{I}}
\newcommand{\ccJ}{\mathcal{J}}
\newcommand{\ccK}{\mathcal{K}}
\newcommand{\ccL}{\mathcal{L}}
\newcommand{\ccM}{\mathcal{M}}
\newcommand{\ccN}{\mathcal{N}}
\newcommand{\ccO}{\mathcal{O}}
\newcommand{\ccP}{\mathcal{P}}
\newcommand{\ccQ}{\mathcal{Q}}
\newcommand{\ccR}{\mathcal{R}}
\newcommand{\ccS}{\mathcal{S}}
\newcommand{\ccT}{\mathcal{T}}
\newcommand{\ccU}{\mathcal{U}}
\newcommand{\ccV}{\mathcal{V}}
\newcommand{\ccW}{\mathcal{W}}
\newcommand{\ccX}{\mathcal{X}}
\newcommand{\ccY}{\mathcal{Y}}
\newcommand{\ccZ}{\mathcal{Z}}

%%%%%%%%%%%%%%%%%%%%%%%%%%%%%%%
% Bold font for greek letters %
%%%%%%%%%%%%%%%%%%%%%%%%%%%%%%%
\newcommand{\bfalpha}{\bm{\alpha}}
\newcommand{\bfbeta}{\bm{\beta}}
\newcommand{\bfgamma}{\bm{\gamma}}
\newcommand{\bfGamma}{\bm{\Gamma}}
\newcommand{\bfdelta}{\bm{\delta}}
\newcommand{\bfDelta}{\bm{\Delta}}
\newcommand{\bfeps}{\bm{\varepsilon}}
\newcommand{\bfzeta}{\bm{\zeta}}
\newcommand{\bfeta}{\bm{\eta}}
\newcommand{\bftheta}{\bm{\theta}}
\newcommand{\bfTheta}{\bm{\Theta}}
\newcommand{\bfiota}{\bm{\iota}}
\newcommand{\bfkappa}{\bm{\kappa}}
\newcommand{\bflambda}{\bm{\lambda}}
\newcommand{\bfLambda}{\bm{\Lambda}}
\newcommand{\bfmu}{\bm{\mu}}
\newcommand{\bfnu}{\bm{\nu}}
\newcommand{\bfxi}{\bm{\xi}}
\newcommand{\bfXi}{\bm{\Xi}}
\newcommand{\bfpi}{\bm{\pi}}
\newcommand{\bfPi}{\bm{\Pi}}
\newcommand{\bfrho}{\bm{\rho}}
\newcommand{\bfsigma}{\bm{\sigma}}
\newcommand{\bfSigma}{\bm{\Sigma}}
\newcommand{\bftau}{\bm{\tau}}
\newcommand{\bfupsilon}{\bm{\upsilon}}
\newcommand{\bfUpsilon}{\bm{\Upsilon}}
\newcommand{\bffy}{\bm{\varphi}}
\newcommand{\bfFy}{\bm{\Phi}}
\newcommand{\bfchi}{\bm{\chi}}
\newcommand{\bfpsi}{\bm{\psi}}
\newcommand{\bfPsi}{\bm{\Psi}}
\newcommand{\bfomega}{\bm{\omega}}
\newcommand{\bfOmega}{\bm{\Omega}}

%%%%%%%%%%%%%%%%%%%%%%%%%%%%%%%%%%%
% Shortcut for some greek letters %
%%%%%%%%%%%%%%%%%%%%%%%%%%%%%%%%%%%
\newcommand{\eps}{\varepsilon}      % Epsilon
\newcommand{\fy}{\varphi}           % phi
\newcommand{\Fy}{\Phi}

%%%%%%%%%%%%%%%%%%%%%%%%%%%
% Mathematical notations  %
%%%%%%%%%%%%%%%%%%%%%%%%%%%
% DeclareMathOperator: recommended for well-defined mathematical operators
% DeclareMathOperator*: supports underneath with limits
% newcommand: recommended for lazy short-cut of frequently used expressions or operators needs math mode font %%%%%%

% Complex numbers
\DeclareMathOperator{\RE}{Re}
\DeclareMathOperator{\IM}{Im}
% Linear algebra
\DeclareMathOperator{\Tr}{Tr}
\DeclareMathOperator{\GL}{GL}
\DeclareMathOperator{\rank}{rank}
% Probability
\DeclareMathOperator{\p}{P}  % Probability
\DeclareMathOperator{\Var}{Var}  % Variance
\DeclareMathOperator{\Cov}{Cov}  % Covariance
% Optimization
\DeclareMathOperator*{\argmin}{\arg\!\min}  % with star: under subscript
\DeclareMathOperator*{\argmax}{\arg\!\max}

% Short-cut expression
\newcommand{\DKL}{D_{\mathrm{KL}}}  % KL divergence
\newcommand{\DTV}{D_{\mathrm{TV}}}  % Total variation distance
\newcommand{\Df}{D_f}  % f-divergence
\newcommand{\Dalpha}{D_\alpha}  % alpha-divergence
\newcommand{\grad}{\nabla}  % gradient
\newcommand{\dd}{\mathop{}\!\mathrm{d}}     % derivative
\newcommand{\pdd}{\mathop{}\!\partial}      % Partial derivative

%% Note: add \left, \right automatically adjust the size, e.g. when use \frac{}{}
% Set
\newcommand{\set}[1]{\left\{ {#1} \right\}}
% Ceiling
\newcommand{\ceil}[1]{\left\lceil {#1} \right\rceil}
% Flooring
\newcommand{\floor}[1]{\left\lfloor {#1} \right\rfloor}
% norm
\newcommand{\norm}[1]{\left\lVert {#1} \right\rVert}
% absolute value
\newcommand{\abs}[1]{\left\lvert {#1} \right\rvert}
% Auto-scale parentheses: useful in equations
\newcommand{\paren}[1]{ \left( {#1} \right) }
% Auto-scale square brackets: useful in equations
\newcommand{\sbrak}[1]{ \left[ {#1} \right] }
% dot product
\newcommand{\dotp}[2]{ {#1} \cdot {#2} }
% inner product
\newcommand{\inner}[2]{\left\langle {#1}, {#2} \right\rangle}
% outer product
\newcommand{\outerp}[1]{ {#1} {#1}^T }  % \outer already pre-defined
\newcommand{\outerpp}[1]{ \left( {#1} \right) \left( {#1} \right)^T }
% trace
\newcommand{\tr}[1]{\Tr \left( {#1} \right)}
% expectation
\newcommand{\Exp}[2]{\mathbb{E}_{{#2}} \left[  {#1} \right]}
% conditional expectation: auto-size of verticle line by \middle
\newcommand{\CondExp}[3]{\mathbb{E}_{{#3}} \left[  {#1} \middle| {#2} \right]}
% KL divergence
\newcommand{\KL}[2]{D_{\mathrm{KL}} \left( {#1} \| {#2} \right)}
% max KL divergence
\newcommand{\KLmax}[2]{D_{\mathrm{KL}}^{\max} \left( {#1} \| {#2} \right)}
% Minimization, maximization, subject to
\newcommand{\minimize}[2]{ \underset{ {#2} }{\text{minimize }} {#1} }
\newcommand{\maximize}[2]{ \underset{ {#2} }{\text{maximize }} {#1} }
\newcommand{\subto}[1]{ \text{subject to } {#1} }
% Fraction of derivatives and partial derivatives
\newcommand{\ddfrac}[2]{\frac{ \mathrm{d}{#1} }{ \mathrm{d}{#2} }}
\newcommand{\ddfracc}[1]{\frac{ \mathrm{d} }{ \mathrm{d}{#1} }}
\newcommand{\pfrac}[2]{\frac{ \partial{#1} }{ \partial{#2} }}
\newcommand{\pfracc}[1]{\frac{\partial}{ \partial{#1} }}
% Average of summation
\newcommand{\avgsum}[3]{\frac{1}{ {#3} } \sum_{ {#1}={#2} }^{ {#3} }}
% Indicator function
\newcommand{\indicator}[1]{\mathbbm{1} \left( {#1} \right) }
% Quadratic form
\newcommand{\quaform}[2]{ {#1}^T {#2} {#1}}
\newcommand{\quaforms}[2]{\left( {#1} \right)^T {#2} \left( {#1} \right)}

%%%%%%%%%%%%%%%%%%%%%%%%%%%%%%%%%%%%%%%
% Paper-specific shortcuts as follows %
%%%%%%%%%%%%%%%%%%%%%%%%%%%%%%%%%%%%%%%
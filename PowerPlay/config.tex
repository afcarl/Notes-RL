%%%%%%%%%%%%%%%%%%%%%%%%%
% Version: Feb 28, 2018 %
%%%%%%%%%%%%%%%%%%%%%%%%%


% Basic setting
\usepackage[utf8]{inputenc} % allow utf-8 input
\usepackage[T1]{fontenc}    % use 8-bit T1 fonts
\usepackage{hyperref}       % hyperlinks
\usepackage{url}            % simple URL typesetting
\usepackage{nicefrac}       % compact symbols for 1/2, etc.
\usepackage{microtype}      % microtypography
\usepackage{authblk}        % Authors and affiliations


% Bibliography management (get item from Google Scholar)
% Use: 
%   1. Create a file called 'references.bib'
%   2. Copy items from Google scholar
%   3. \cite{tamar2016value}
%   4. \printbibliography
\usepackage[
style=numeric,      % [numeric, alphabetic]
sorting=nty         % nty: name, title, year
]{biblatex}
\addbibresource{references.bib}

% Figure, tables
\usepackage{graphicx}
\graphicspath{ {img/} }     % search path of images
\usepackage{subcaption}     % Multiple figures \subfigure
\usepackage{booktabs}       % professional-quality tables


% Mathematical
\usepackage{amsmath, amssymb, amsthm, amsfonts}
\usepackage{bbm}                    % 'mathbb' for digits
\usepackage[ruled]{algorithm2e}     % Algorithms package


% Define useful commands
\newtheorem{definition}{Definition}[section]  % [section]: recount after each section
\newtheorem{theorem}{Theorem}[section]
\newtheorem{lemma}[theorem]{Lemma}            % [theorem]: same counter as theorem
\newtheorem{corollary}[theorem]{Corollary}
\newtheorem{proposition}[theorem]{Proposition}

\newtheorem{assumption}{Assumption}
\newtheorem{question}{Question}
\newtheorem{remark}{Remark}
\newtheorem{case}{Case}

\DeclareMathOperator*{\argmin}{\arg\!\min}
\DeclareMathOperator*{\argmax}{\arg\!\max}


% Math bold font
\newcommand{\x}{\mathbf{x}}
\newcommand{\X}{\mathbf{X}}


% Blackboard bold: e.g. real/complex set
\newcommand{\bN}{\mathbb{N}}  % Natural set
\newcommand{\bR}{\mathbb{R}}  % real set
\newcommand{\bC}{\mathbb{C}}  % complex set
\newcommand{\bE}{\mathbb{E}}  % expectation operator


% Calligraphic font: e.g. relations, power set, topology space
\newcommand{\cA}{\mathcal{A}}
\newcommand{\cB}{\mathcal{B}}
\newcommand{\cS}{\mathcal{S}}
\newcommand{\cF}{\mathcal{F}}
\newcommand{\cU}{\mathcal{U}}


% Commonly used math notations
\newcommand{\Var}{\mathrm{Var}}  % Variance
\newcommand{\Cov}{\mathrm{Cov}}  % Covariance
\newcommand{\T}{\top}            % better for transpose operation
%% Note: add \left and \right to automatically adjust the size, e.g. fraction arguments
% norm
\newcommand{\norm}[1]{\left\lVert {#1} \right\rVert}
% absolute value
\newcommand{\abs}[1]{\left\lvert {#1} \right\rvert}
% inner product
\newcommand{\inner}[2]{\left\langle {#1}, {#2} \right\rangle}
% expectation
\newcommand{\Exp}[2]{\mathbb{E}_{{#2}} \left[  {#1} \right]}
% conditional expectation
\newcommand{\CondExp}[3]{\mathbb{E}_{{#3}} \left[  {#1} \middle| {#2} \right]}
% KL divergence
\newcommand{\KL}[2]{D_{\mathrm{KL}} \left( {#1} \| {#2} \right)}
% max KL divergence
\newcommand{\KLmax}[2]{D_{\mathrm{KL}}^{\max} \left( {#1} \| {#2} \right)}


%%%%%%%%%%%%%%%%%%%%%%%%%%%%%%%%%%%%%%%
% Paper-specific shortcuts as follows %
%%%%%%%%%%%%%%%%%%%%%%%%%%%%%%%%%%%%%%%
